\documentclass[a4paper, 11pt]{article}
\usepackage{amsmath}
\begin{document}
\title{Exercise 1\\{\Large Kernel Methods in Machine Learning}}
\maketitle
\section{Question 1}
\subsection{}
\begin{itemize}
\item $k_1(\mathbf{x}, \mathbf{y})=c$ is a kernel, given $\forall \mathbf{x}, \phi(\mathbf{x})=\sqrt{c}$.
\item $k_2(\mathbf{x}, \mathbf{y})=\langle\mathbf{x}, \mathbf{y}\rangle$ is a kernel, given $\forall \mathbf{x}, \phi(\mathbf{x})=\mathbf{x}$.
\item According to the rule that conic sum of kernel is a kernel, $k_3(\mathbf{x}, \mathbf{y})=\langle\mathbf{x}, \mathbf{y}\rangle+c$ is a kernel.
\item According to the rule that product of kernel functions is a kernel, $K(\mathbf{x},\mathbf{y})=(\langle\mathbf{x}, \mathbf{y}\rangle+c)^m$ is a kernel.
\end{itemize}
\subsection{}
The expansion of $K(\mathbf{x}, \mathbf{y}) =( \langle \mathbf{x}, \mathbf{y} \rangle)^3$ is:\\
\begin{align*}
K(\mathbf{x}, \mathbf{y})  = x_1y_1x_1y_1x_1y_1+x_1y_1x_2y_2x_1y_1+x_2y_2x_1y_1x_1y_1+
x_2y_2x_2y_2x_1y_1+\\x_1y_1x_1y_1x_2y_2+x_1y_1x_2y_2x_2y_2+
x_2y_2x_1y_1x_2y_2+x_2y_2x_2y_2x_2y_2
\end{align*}
So we get:
\begin{align*}
&\phi(\mathbf{x}) = (x_1x_1x_1,x_1x_2x_1,x_2x_1x_1,x_2x_2x_1,x_1x_1x_2,x_1x_2x_2,x_2x_1x_2,x_2x_2x_2)\\
&\phi(\mathbf{y}) = (y_1y_1y_1,y_1y_2y_1,y_2y_1y_1,y_2y_2y_1,y_1y_1y_2,y_1y_2y_2,y_2y_1y_2,y_2y_2y_2)\\
\end{align*}
\end{document}